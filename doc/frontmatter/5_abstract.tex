% called by main.tex
%

\section*{Abstract}
\addcontentsline{toc}{section}{Abstract}
\label{sec::abstract}

Analysing how a newborn baby reacts to different stimuli is crucial to diagnose possible neurological conditions. Traditionally, these assessments have been performed manually by health professionals, which to some extent leads to subjectivity, and possible misdiagnosis. In addition, such assessments require the immediate availability of a specialist, which in critical situations can delay assessments, putting the baby's health at risk.

This project uses advanced tools such as \textbf{Deep Learning}, and \textbf{Machine Learning} to automate, and improve the detection, and \textbf{classification of newborn cries} in order to detect possible pathologies or diseases. Artificial Intelligence analyses of newborn videos, provides a consistent, and accurate assessment of newborn responses to stimuli can be achieved, accelerating the diagnosis of a possible disease.  

Specifically, this study focuses on the detection of Hypoxic-Ischaemic Encephalopathy (\textbf{HIE}) by assessing newborn crying in response to nociceptive stimuli. To achieve this goal, several audio analysis approaches have been applied, with the project focusing on the employing of Machine Learning models, trained on labelled cry data, which have demonstrated high accuracy in the recognition of cry patterns.

This study explores feature extraction techniques such as Mel-Frequency Cepstral Coefficients (\textbf{MFCC}), and Linear Predictive Coding (\textbf{LPC}), and has implemented models such as Multilayer Perceptron (\textbf{MLP}), Support Vector Machine (\textbf{SVM}), and Long Short-Term Memory (\textbf{LSTM}). Promising results have been achieved with these models, with accuracies of up to 90\%.

The above methods not only help to detect HIE, but also open up the possibility of diagnosing other health problems through cry analysis. This research thus highlights the potential of machine learning to improve paediatric diagnosis, and neonatal care.

\vspace{\baselineskip}
\textbf{Keywords}: Deep Learning, Machine Learning, Audio analysis, Classification of crying, Newborns, Mel Frequency Cepstral Coefficients (MFCC), Linear predictive coding (LPC), Multilayer Perceptron (MLP), Support Vector Machine (SVM), Long Short-Term Memory (LSTM), Hypoxic Ischaemic Encephalopathy


\newpage
\section*{Overview}
\addcontentsline{toc}{section}{Overview}
\label{sec::resumen}

Analysing how a neonate reacts to a stimulus is crucial for a proper diagnosis and treatment of the possible neurological disease it may be suffering from. Historically, such assessments have been performed manually by healthcare staff, basing the classification of the stimulus on their judgement, and interpretation of the reactions observed in the newborn. This approach can sometimes present problems. On the one hand, the assessment can be subjective, and vary among different health specialists, which can lead to erroneous diagnoses. On the other hand, as these are situations that require rapid action, if at the critical moment there is no doctor or specialist available for this specific disease, the assessment of the newborn can be complicated or delayed, putting his or her health at risk. 

To solve these problems, advanced tools are now available, such as \textbf{Deep Learning} and \textbf{Machine Learning}, which have the main feature of being able to automate, and improve processes such as the detection, and classification of certain diseases. Through the collection of videos of newborns, and Deep Learning techniques with which large volumes of data can be analysed, a concise, and uniform assessment of the state of health of babies can be achieved. This avoids inconsistencies in classifications, and increases the accuracy, and speed of diagnosis.

This study aims to provide new knowledge for the detection of \textbf{Hypoxic Ischaemic Encephalopathy (HIE)} by assessing only the sound stimuli that a newborn generates. Several \textbf{audio analysis} approaches can be applied to solve this problem, such as a spectral analysis in which the sound is decomposed to analyse its frequency components. An extraction of temporal features that may include the duration of the cry, the intervals between cries or the intensity patterns can be carried out. Convolutional Neural Networks (CNNs) can also be used to classify the different types of cries according to the audio characteristics and, Deep Learning, and Machine Learning models can also be applied. With proper training, these models may be able to recognise complex patterns in audio data, and \textbf{classify different types of crying} with good accuracy based on labelled crying datasets.

Therefore, the purpose of this study is to explore the ability of Machine Learning and Deep Learning to identify, and classify the sound responses produced by a neonate to a stimulus by analysing its cry. To achieve this goal, we have opted for feature extraction using techniques such as Mel Frequency Cepstral Coefficients (\textbf{MFCC}) or Linear Predictive Coding (\textbf{LPC}) and then create different models such as Multilayer Perceptron (\textbf{MLP}), Support Vector Machine (\textbf{SVM}), and Long Short-Term Memory (\textbf{LSTM}). These models were trained first with a public dataset, to see if the results were promising. Subsequently, these same models have been trained with project-specific audios, thus achieving an accurate classification of the different sounds that can be found present in the audios available for this study, with accuracies of up to \textbf{90\%}. 

This approach not only facilitates the detection of HIE, but also leaves the possibility that this technique can be applied to the identification of other health conditions through the response to crying. In this way this research presents a new avenue for paediatric diagnosis with the help of innovative techniques, and the importance of machine learning in improving neonatal disease detection and care.

\vspace{\baselineskip}
\textbf{Keywords}: Deep Learning, Machine Learning, Audio analysis, Classification of crying, Newborns, Mel Frequency Cepstral Coefficients (MFCC), Linear predictive coding (LPC), Multilayer Perceptron (MLP), Support Vector Machine (SVM), Long Short-Term Memory (LSTM), Hypoxic Ischaemic Encephalopathy


