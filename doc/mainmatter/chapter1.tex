% called by main.tex
%
\chapter{Introduction}
\label{ch::chapter1}

\section{Problem overview and relevance}
Crying is the main way babies communicate in their first months of life, so it is one of the ways they express their needs, emotional states or discomfort. For this reason, it is very important to learn to understand what babies are trying to express in order to be able to react, and respond effectively to their requests. This is the main basis for an accurate detection of \textbf{Hypoxic Ischemic Encephalopathy} (HIE) in newborns and if positive, to what degree. This encephalopathy is one of the most common and is caused by a lack of oxygen to the baby’s brain at some point during pregnancy or delivery \cite{riley_neonatal}. Its detection is complex, and for a complete and accurate diagnosis, the following tests, and examinations must be performed on the newborn to assess its state of health:
\begin{itemize}
    \item Different scoring scales, an Apgar\footnote{Rapid test performed at one and five minutes after the birth of the baby.\cite{medlineplus_apgar}} test which assesses different factors in the development of any newborn and a Sarnat\footnote{Clinical tool used to assess the severity of neonatal encephalopathy.\cite{modified_sarnat_score}} test which is an assessment of neonatal encephalopathy.
    \item Diagnostic tests, such as monitoring of brain function, magnetic resonance imaging (MRI) to detect possible brain damage, blood tests to assess possible imbalances that may be indicative of hypoxia.
    \item Monitoring of vital signs, such as heart rate, respiration, and oxygenation to detect any warning signs.
    \item Clinical assessments, evaluating the circumstances of delivery, and a \textbf{physical and neurological examination of the newborn}. This examination assesses the newborn's muscle tone, reflexes, and level of awareness to the stimuli that the health care provider makes on it \cite{clarke_management_2017}.
\end{itemize}

As can be seen, the detection of this condition is very broad, so any contribution to this field can be of great help. Giving special relevance to clinical evaluations, as mentioned above, it is very important to observe the baby's reactions to stimuli in the first hours of life. According to the Garcia-Alix score scale \cite{garcia_alix_scale}, there are multiple aspects that must be taken into account to determine the presence of HIE in a newborn. However, García-Alíx and other authors agree that alertness is one of the most important ~\cite{kurinczuk_review,sarnat_scale}. To assess this, it was decided to observe \textbf{four characteristics} in each of the files that make up the dataset, and to determine the newborn's response to a stimulus: mouth, and eye opening, frowning, and crying.

This study focuses on the assessment of crying, so it should be noted that not every cry reflects a healthy state of health in the infant. A cry that is considered indicative of a healthy neurological state must meet certain characteristics: it must be sustained, and last at least eight seconds, it must not be abruptly interrupted, and it must exhibit a progression and modulation in pitch that indicates normal sensory and emotional responsiveness.

Advanced audio processing, and machine learning techniques have been used in numerous studies to classify, and interpret these complex sounds that may indicate abnormal conditions in the newborn, and therefore require immediate medical attention \cite{hammoud_machine_2024, tuduce_why_2018}. This has improved non-invasive infant monitoring, and provided caregivers with valuable tools for more informed, and comprehensive care of newborns with these conditions. 

Putting these novel advances into practice, one experimental approach to address the challenge of data acquisition in this sector is that proposed in the NeoCam project by the University of Cadiz \cite{ruiz_zafra_neocam_2023}. This project proposes a strategy of placing automatic cameras in incubators to monitor the vital signs of the newborn, and interpret possible problems or ailments through an analysis of its movements, facial expressions, respiratory rhythm, sleep cycles etc. With this strategy, automatic data acquisition, and even real-time assessment of crying could be achieved, which would be very beneficial for the early \textbf{detection of HIE}. 


\section{Assumptions and limitations}
This project, like others that focus on \textbf{audio classification} or \textbf{stimulus assessment by infants}, is based on several key assumptions. Firstly, that the data used is of good quality and, above all, representative of the different types of crying there is. In this particular study, it was observed that the available audios were quite pertinent as they included \textbf{recordings of different babies in various health states}. However, this was also a limitation because \textbf{all the available data was from the same region}, specifically from the same hospital, which introduced some selection bias. This consequentially resulted into a likelihood of similar crying patterns. This is because if the data used to train the algorithm is more representative of some groups of people than others, the predictions from the model will be systematically worse for unrepresented or under-representative groups leading to inaccuracies. 


Another assumption that has been made in this study is in relation to the conditions under which the data have been obtained, i.e. that the \textbf{recordings are free of interference and large or loud ambient noises}, which could distort or mask the relevant sounds. In this study, data provided was from a maternity ward of a hospital, a place that does not usually have much background noise other than that of other babies or health care staff. However, there is an intrinsic limitation as, the model is fed with mostly clean acoustic signals, therefore biasing the predictions when the data acquired from noisier environments is utilized. 

It was also assumed that all the datasets worked with in this project were \textbf{sufficient to extract enough acoustic features needed by the models}. To increase the value of this assumption, two different types of feature extraction were used, Mel frequency cepstral coefficients (\textbf{MFCC}) on the one hand, and linear prediction features (\textbf{LPC}) on the other hand. By comparing the results produced by different models with both techniques, a consistent and accurate classification on the target dataset has been achieved.

Another limitation that stands out in this project as in any other that applies Machine Learning techniques, both when classifying audio and other types of data, is the limitation known as \textit{black boxes}\cite{rocha_how_2023}. This problem refers to the opacity of the models on how they reach certain conclusions, therefore leading to improper usage or application by staff hence resulting into biased results. This issue generates difficulties with implementation, and mistrust by health personnel and patients.

Finally, one of the main limitations of the project is the \textbf{number of samples} (audios) provided. The collection of health data is always an extremely sensitive issue, firstly, it requires consent from the patient and, secondly, there is an ethical duty of researchers to use this data correctly to avoid misuse. Many studies have expressed concerns about \textbf{data privacy} \cite{stewart_empirical_2002} and the challenge of \textbf{informed consent} \cite{dickson_enduring_2015} affecting the collection and use of clinical data \cite{whitley_consent_2012}. In this case, the purpose of the study is to assess crying in preterm infants, an aspect that could not be predicted, making it difficult to obtain parental consent for newborn recordings. Without knowing the baby’s health condition at the time of birth, many parents did not want to give consent for their baby to be videotaped, which made it difficult to obtain data for this type of study. This therefore has affected the representativeness and generalisability of the results.


\section{Ethical and Social Impact}
This section is directly related to the previous section, as the assumptions we have made about this project may lead to limitations that impact on the technical feasibility of the project, and also have ethical, and social implications. 

As mentioned above, generalization is one of the possible issues faced when there is a lack of diversity within the datasets. Numerous studies have reported the challenges presented by \textbf{MLP}, \textbf{SVM} and \textbf{LSTM} models among others, highlighting the challenges associated with generalising results to different settings \cite{Rahimzad2021, Oukhouya2023, Lakshminarayanan2019}. It is for this reason that poor generalisation can lead to incorrect diagnoses, according to the article \textit{Key challenges for delivering clinical impact with artificial intelligence} by Sendak, M. P., D'Arcy, J., & Ratwani, R. M. \cite{Kelly2021}. This article proposed improving transparency and diversity in the datasets, and especially the need to implement these technologies in clinical practice to overcome generalisation issues. 

Another aspect that has also been discussed above is the lack of trust in the models. The loss or scarcity of public trust in systems that apply Machine Learning for disease detection presents a major ethical challenge. If the systems present problems in disease analysis that can lead to amplified human error, this leads to health professionals showing distrust and resignation to using this type of technology \cite{Murphy2021}. In the case of newborns presenting with HIE early in life, therapeutic hypothermia initiated within 6 hours of postnatal life significantly reduces mortality and major neurodevelopmental disability \cite{Tagin2012}. However, if this treatment is applied incorrectly to neonates who do not meet the criteria for this treatment, it can lead to unnecessary, and potentially serious complications.

Of course, we must highlight the importance of privacy, and data security, according to the \textit{General Data Protection Regulation} (GDPR) \cite{EU2016} Sensitive data is that data which reveals information relating to the health or origin of the patient, their genetic information, etc. In this project, both videos and audios of newborns are considered sensitive data, and therefore require special treatment to ensure privacy, and robust measures to ensure security. 

Finally, a problem raised by this type of study in relation to the ethical and social impact is the responsibility or culpability in case of error: who is responsible for an erroneous diagnosis guided by a disease prediction system? Should it be the doctor, who followed these recommendations? Should it be the programmers who implemented the system? Or perhaps the hospital or company that implemented this technology? \cite{Naik2022, Griffin2021, MolnarGabor2020}. In the scenario where a machine learning-based system erroneously predicts that a neonate is suffering from HIE, and the treatment causes harm to the patient, the responsibility for who should be held accountable for this harm can be complex.


\section{Personal motivation}
My personal motivation for this project stems from a combination of several factors. Firstly, since my first steps at the University of Burgos, my interest in the field of health has been intertwined with the value and importance I placed on acquiring knowledge in computer science. Thus, my first steps in the world of Machine Learning began within the \myurl{https://admirable-ubu.es/}{ADMIRABLE group} of the University of Burgos with my final degree project dedicated to the \myurl{https://github.com/lnc1002/TFG-Evaluacion-Ejercicios-Rehabilitacion.git}{analysis and classification of movements of people with Parkinson's} . This project not only marked my introduction to the field of artificial intelligence, but also reaffirmed my interest in continuing to carry out projects in which technological solutions are used to improve the quality of people’s life.
 
Working on this project allowed me to learn first-hand how technology can improve the diagnosis and treatment of diseases, and my interest in this field grew until I found this new project offered by the \myurl{https://gicap.ubu.es/main/home.shtml}{Applied Computational Intelligence (GICAP)} group at the University of Burgos. What has motivated me the most in this project is being able to combine my programming skills with my taste for solving problems with a direct impact on the health and well-being of people, especially in the case of such small patients in whom an early diagnosis can help their quality of life from a very early stage.
 
This project has also allowed me to put into practice much of the knowledge acquired in the master's degree in Biomedical Data Science at the Rovira and Virgili University and expand my skills in biomedical data analysis. In addition, it has been an excellent opportunity to explore areas such as supervised or unsupervised learning and working with sensitive data while ensuring privacy and information security. This challenge has not only brought new knowledge and value to my professional experience but has also reinforced my view on the importance of developing technology in an ethical and responsible manner.
 
Looking to the future, I am motivated to continue working in the field of health data analytics because advances in this field can be translated into direct improvements for the well-being of the population, including the global population, by making them more accessible and efficient.